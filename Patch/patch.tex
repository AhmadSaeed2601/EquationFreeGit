% input *.m files for ... AJR, Nov 2017
%!TEX root = ../Doc/equationFreeDoc.tex
\section{Patch scheme for given microscale discrete space system}
\label{sec:patch}
\localtableofcontents

The patch scheme applies to spatio-temporal systems where
the spatial domain is larger than what can be computed in
reasonable time. Then one may simulate only on small patches
of the space-time domain, and produce correct macroscale
predictions by craftily coupling the patches across
unsimulated space \cite[e.g.]{Hyman2005, Samaey03b,
Samaey04, Roberts06d, Liu2015}.

The spatial discrete system is to be on a lattice such as
obtained from finite difference approximation of a \pde.
Usually continuous in time.

\paragraph{Quick start}
For an example, see \cref{sec:configPatches1eg,sec:configPatches2eg}  for basic code that uses the provided functions to simulate Burgers'~\pde\ and a nonlinear `diffusion' \pde.

\input{../Patch/configPatches1.m}
\input{../Patch/patchSmooth1.m}
\input{../Patch/patchEdgeInt1.m}
\input{../Patch/BurgersExample.m}
\input{../Patch/HomogenisationExample.m}
\input{../Patch/waterWaveExample.m}
% 2D stuff
\input{../Patch/configPatches2.m}
\input{../Patch/patchSmooth2.m}
\input{../Patch/patchEdgeInt2.m}
\input{../Patch/wave2D.m}



\begin{body}
\subsection{To do}
\begin{itemize}
\item Testing is so far only qualitative.  Need to be quantitative.
\item Multiple space dimensions.
\item Heterogeneous microscale via averaging regions.
\item Parallel processing versions.
\item ??
\item Adapt to maps in micro-time?  Surely easy, just an example.
\end{itemize}


\subsection{Miscellaneous tests}
\input{../Patch/patchEdgeInt1test.m}
\input{../Patch/patchEdgeInt2test.m}

\end{body}
