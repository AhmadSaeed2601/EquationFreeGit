% input *.m files for the Patch scheme in 1D and 2D. AJR,
% Nov 2017 -- Feb 2019
%!TEX root = ../Doc/eqnFreeDevMan.tex
\chapter{Patch scheme for given microscale discrete space system}
\label{sec:patch}
\localtableofcontents

\section{Introduction}


Consider spatio-temporal multiscale systems where
the spatial domain is so large that a given microscale code cannot be computed in a
reasonable time.  The \emph{patch scheme} computes the microscale details only on small
patches of the space-time domain, and produce correct
macroscale predictions by craftily coupling the patches
across unsimulated space \cite[e.g.]{Hyman2005, Samaey03b,
Samaey04, Roberts06d, Liu2015}.  The resulting macroscale
predictions were generally proved to be consistent with the
microscale dynamics, to some specified order of accuracy, in
a series of papers: 1D-space dissipative systems
\cite[]{Roberts06d, Bunder2013b}; 2D-space dissipative
systems \cite[]{Roberts2011a}; and 1D-space wave-like
systems \cite[]{Cao2014a}.

The microscale spatial structure is to be on a lattice such
as obtained from finite difference\slash element\slash volume approximation of a \pde.
The microscale is either continuous or discrete in time.

\paragraph{Quick start}
See \cref{sec:configPatches1eg,sec:configPatches2eg} which
respectively list example basic code that uses the provided 
functions to simulate the 1D Burgers'~\pde, and a 2D 
nonlinear `diffusion'~\pde.  Then see \cref{fig:constructpatch}.

\begin{figure}
\begin{maxipage}
\setlength{\WD}{0.048\linewidth}%%%%%%%%%%%%%%%%
\centering
\caption{\label{fig:constructpatch}The Patch methods, \cref{sec:patch}, accelerate simulation\slash integration of multiscale systems with interesting spatial\slash network structure\slash patterns. The  methods use your given microsimulators whether coded from \textsc{pde}s, lattice systems, or agent\slash particle microscale simulators.
The patch functions require that a user configure the patches, and interface the coupled patches with a time integrator\slash simulator. 
This chart overviews the main functional recursion involved.
}
\begin{tikzpicture}[node distance = 3ex, auto]
\tikzstyle{bigblock} = [rectangle, draw, thick, text width=20.5\WD, text badly centered, rounded corners, minimum height=4ex]
\tikzstyle{block} = [rectangle, draw=blue!80!black, thick, anchor=west, fill=white,
    text width=10\WD, text ragged, rounded corners, minimum height=8ex]
 \tikzstyle{smallblock} = [rectangle, draw=blue!80!black, thick, anchor=west, fill=white,
    text width=6\WD, text ragged, rounded corners, minimum height=8ex]   
 \tikzstyle{tinyblock} = [rectangle, draw=blue!80!black, thick, anchor=west, fill=white,
    text width=4.5\WD, text ragged, rounded corners, minimum height=8ex]      
\tikzstyle{line} = [draw, -latex']
\tikzstyle{lined} = [draw, latex'-latex']
\node [bigblock,draw=blue!80!black,fill=blue!10] (gaptooth) {\textbf{Patch scheme for spatio-temporal dynamics}\\[1ex]
    \begin{tikzpicture}[node distance = 3ex, auto]
    \node [block] (configPatches) {\textbf{Setup problem and construct patches}
    
    Invoke \texttt{configpatches1} (for 1D) or \texttt{configpatches2} (for 2D) to setup the microscale problem (\textsc{pde}, domain, boundary conditions, etc) and the desired patch structure (number of patches, patch size, coupling order, etc).
These initialise the global struct \texttt{patches} that contains information required to simulate the microscale dynamics within each patch. 
If necessary, define additional components for the struct \texttt{patches} (e.g., see \texttt{homogenisationExample.m}).};

    \node [block, below=of configPatches] (microPDE) {\textbf{Simulate the multiscale system}
    
    Generally invoke a \textsc{pde} integrator to simulate the multiscale system of a user's microscale code within spatially separated patches. This integrator may be \script\ defined (e.g., \texttt{ode15s\slash ode45}) or user defined (e.g., a projective integrator).
    
    Input to the integrator is the function \texttt{patchSmooth1} (for 1D) or \texttt{patchSmooth2} (for 2D) which  interfaces to the microscale's code. Other inputs are the macro-time span and initial conditions. Output from the integrator is the solution field over the given time span, but only within the defined patches.};
    
    \node [smallblock, above right=-3\WD and 2\WD of microPDE] (patchSmooth1) {\textbf{Interface to microscale}
    
    \texttt{patchSmooth1\slash2} interfaces with the microscale \textsc{pde}\slash lattice system and invokes the patch coupling conditions. Input is the field in every patch at one time-step, and output is time-derivatives of the field, or values at the next time-step, as appropriate.};
    
    \node [tinyblock, below left=4ex and -3\WD of patchSmooth1] (coupling) {\textbf{Coupling conditions}
    
    \texttt{patchEdgeInt1\slash2} (for 1D or 2D respectively) couple patches together by setting edge-values via macroscale interpolation of order in global \texttt{patches.ordCC}.};
    \node [tinyblock, below right=4ex and -3\WD of patchSmooth1] (micropde) {\textbf{Microscale system}
    
    A user's microscale code, pointed to by \texttt{patches.fun}, codes the microscale dynamics on the interior of the patch microgrids, \texttt{patches.x}, to compute either a micro-step or time-derivatives.
    };
%    \node [block,draw=red!80!black,fill=red!10, below=of microPDE] (pi) {\hyperref[fig:constructPI]{\textbf{\textbf{Projective integration scheme (if needed)}}}
%    };    
    \path [lined,very thick,-latex] (configPatches) -- (microPDE);
    \path [lined,very thick] (microPDE) to[out=0,in=180] (patchSmooth1);
    \path [lined,very thick] (patchSmooth1) to[out=270,in=90] (coupling);
    \path [lined,very thick] (patchSmooth1) to[out=270,in=90] (micropde);
%    \path [lined,very thick,latex-latex] (microPDE) -- (pi);
    \end{tikzpicture}
    };   
\node [bigblock,draw=black,below=of gaptooth] (process) {\textbf{Process results and plot}};
 \path [lined,very thick,-latex] (gaptooth) -- (process);
\end{tikzpicture}
\end{maxipage}
\end{figure}



\input{../Patch/configPatches1.m}
\input{../Patch/patchSmooth1.m}
\input{../Patch/patchEdgeInt1.m}
\input{../Patch/homogenisationExample.m}
\begin{devMan}
\input{../Patch/BurgersExample.m}
\input{../Patch/ensembleAverageExample.m}
\input{../Patch/waterWaveExample.m}
\end{devMan}

% 2D stuff
\input{../Patch/configPatches2.m}
\input{../Patch/patchSmooth2.m}
\input{../Patch/patchEdgeInt2.m}
\begin{devMan}
\input{../Patch/wave2D.m}



\section{To do}
\begin{itemize}
\item Some users will have microscale that has a fixed microscale lattice spacing, in which case we should code the scale ratio~\(r\) to follow from the choice of the number of lattice points in a patch.
\item More than two space dimensions?
\item Heterogeneous microscale via averaging regions---but I suspect should be separated from simple homogenisation
\item Parallel processing versions.
\item Adapt to maps in micro-time?  Surely easy, just an example.
\end{itemize}


\section{Miscellaneous tests}
\input{../Patch/patchEdgeInt1test.m}
\input{../Patch/patchEdgeInt2test.m}

\end{devMan}
