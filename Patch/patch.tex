% input *.m files for ... AJR, Nov 2017
%!TEX root = ../equationFreeDoc.tex
\section{Patch scheme for given microscale discrete space system}
\label{sec:patch}
\secttoc

The patch scheme applies to spatio-temporal systems where the spatial domain is larger than what can be computed in reasonable time.
Then one may simulate only on small patches of the space-time domain, and produce correct macroscale predictions by craftily coupling the patches across unsimulated space \cite[e.g.]{Hyman2005, Samaey03b, Samaey04, Roberts06d, Liu2015}.

The spatial discrete system is to be on a lattice such as obtained from finite difference approximation of a \pde.
Usually continuous in time.

\input{Patch/patchSmooth1.m}
\input{Patch/makePatches.m}
\input{Patch/patchEdgeInt1.m}
\input{Patch/patchEdgeInt1test.m}
\input{Patch/BurgersExample.m}
\input{Patch/HomogenisationExample.m}
\input{Patch/waterWaveExample.m}

\subsection{To do}
\begin{itemize}
\item Testing is so far only qualitative.  Need to be quantitative.
\item Multiple space dimensions.
\item Heterogeneous microscale via averaging regions.
\item Parallel processing versions.
\item ??
\item Adapt to maps in micro-time?
\end{itemize}
