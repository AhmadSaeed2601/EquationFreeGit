%Initially by AJR, Apr 2017 -- Mar 2019
%!TEX root = eqnFreeDevMan.tex
\chapter{Create, document and test algorithms}
\label{sec:contribute}
\secttoc
\def\LaTeX{LaTeX}% for some unknown reason we need this!! 2018-12-22

For developers to create and document the various functions, we use an idea due to Neil~D. Lawrence of the University of Sheffield.

\begin{itemize}
\item Each class of toolbox functions is located in separate directories in the repository, say~\verb|Dir|.

\item Create a \LaTeX\ file~\verb|Dir/funs.tex|: establish as one \LaTeX\ section that \verb|\input{Dir/*.m}|s the  files of the functions in the class, example scripts of use, and possibly test scripts, \autoref{tbl:funtex}.

\item Each such \verb|Dir/funs.tex| file is to be included from the main \LaTeX\ file \verb|Doc/eqnFreeDevMan.tex| so that people can most easily work on one section at a time: 
\begin{itemize}
\item put \verb|\include{funs}| into \verb|Doc/eqnFreeDevMan.tex|;
\item to include we have to use a soft link so at the command line in the directory~\verb|Doc| execute \verb|ln -s ../Dir/funs.tex|
\footnote{Such soft links are necessary for at least my Mac \textsc{osx} and hopefully work for other developers.  Further, auxiliary files are advantageously also located in the \texttt{Doc} directory.}
\end{itemize}

\item Each toolbox function is documented as a separate section, with tests and examples as separate sections.

\item Each function-section and test-section is to be created as a \script\ \verb|Dir/*.m| file, say \verb|Dir/fun1.m|, so that users simply invoke the function in \script\ as usual by \verb|fun1(...)|.

Some editors may need to be told that \verb|fun1.m| is a \LaTeX\ file.  For example, TexShop on the Mac requires one to execute in a Terminal
\begin{verbatim}
defaults write TeXShop OtherTeXExtensions -array-add "m"
\end{verbatim}

\item \autoref{tbl:format} gives the template for the \verb|Dir/*.m| function-sections.
The format for a example\slash test-section is similar.

\item Any figures from examples should be generated and then saved for later inclusion with the following (which finally works properly for \textsc{Matlab} 2017+)
\begin{verbatim}
set(gcf,'PaperPosition',[0 0 14 10]);% cm
print('-depsc2','Figs/filename')
\end{verbatim}
Include with (do \emph{not} postfix with \verb|.eps| or \verb|.pdf|)
\begin{verbatim}
\includegraphics[scale=0.9]{../Dir/Figs/filename}
\end{verbatim}


\item For every function, generally include at the start of the function a simple example of its use.  The example is only to be executed when the function is invoked with no input arguments.

When appropriate, if a function is invoked with no output arguments, then draw some reasonable graph of the results.


\item In all \script\ code, prefer camal case for variable names (not underscores).


\item In documentation \cite[e.g.,][Ch.~4]{Higham98}: 
write actively, not passive (e.g., avoid ``--tion'' words);
avoid wishy-washy ``can'';
use the present tense;
cross-reference precisely;
and so on.

\end{itemize}


\begin{table}
\caption{\label{tbl:funtex}example \texttt{Dir/*.tex} file to typeset in the master document a function-section, say \texttt{fun.m}, and maybe the test\slash example-sections.}
\VerbatimInput[numbers=left]{../chapterTemplate.tex}
\end{table}
\begin{table}
\caption{\label{tbl:format}template for a function-section \texttt{Dir/*.m} file.}
\VerbatimInput[numbers=left]{../functionTemplate.m}
\end{table}


