%Initiated by AJR, Apr 2017
%!TEX root = equationFreeDoc.tex
\section{Aspects of developing a `toolbox' for patch dynamics}
\secttoc

This appendix documents sketchy further thoughts on aspects of the development.


\subsection{Macroscale grid}

The patches are to be distributed on a macroscale grid: the \(j\)th~patch `centred' at position~\(\Xv_j\in\XX\).
In principle the patches could move, but let's keep them fixed in the first version.
The simplest macroscale grid will be rectangular (\texttt{meshgrid}), but we plan to allow a deformed grid to secondly cater for boundary fitting to quite general domain shapes~\XX.
And plan to later allow for more general interconnect networks for more topologies in application.




\subsection{Macroscale field variables}

The researcher\slash practitioner has to know an appropriate set of macroscale field variables~\(\Uv(t)\in\RR^{d_{\Uv}}\) for each patch.  
For example, first they might be a simple average over a core of a patch of all of the micro-field variables; second, they might be a subset of the average micro-field variables; and third in general the macro-variables might be a nonlinear function of the micro-field variables (such as temperature is the average speed squared).
The core might be just one point, or a sizeable fraction of the patch.

The mapping from microscale variable to macroscale variables is often termed the restriction.

In practice, users may not choose an appropriate set of macro-variables, so will eventually need to code some diagnostic to indicate a failure of the assumed closure.




\subsection{Boundary and coupling conditions}

The physical domain boundary conditions are distinct from the conditions coupling the patches together.
Start with physical boundary conditions of periodicity in the macroscale.

Second, assume the physical boundary conditions are that the macro-variables are known at macroscale grid points around the boundary.  
Then the issue is to adjust the interpolation to cater for the boundary presence and shape.
The coupling conditions for the patches should cater for the range of Robin-like boundary conditions, from Dirichlet to Neumann.
Two possibilities arise: direct imposition of the coupling action \citep{Roberts06d}, or control by the action.

Third, assume that some of the patches have some edges coincident with the boundary of the macroscale domain~\XX, and it is on these edges that macroscale physical boundary conditions are applied.
Then the interpolation from the core of these edge patches is the same as the second case of prescribed boundary macro-variables.
An issue is that each boundary patch should be big enough to cater for any spatial boundary layers transitioning from the applied boundary condition to the interior slow evolution.

Alternatively, we might have the physical boundary condition constrain the interpolation between patches.

Often microscale simulations are easiest to write when `periodic' in microscale space.  
To cater for this we should also allow a control at perhaps the quartiles of a micro-periodic simulator.





\subsection{Mesotime communication}

Since communication limits large scale parallelism, a first step in reducing communication will be to implement only updating the coupling conditions when necessary.
Error analysis indicates that updating on times longer the microscale times and shorter than the macroscale times can be effective
\citep{Bunder2015a}.
Implementations can communicate one or more derivatives in time, as well as macroscale variables.

At this stage we can effectively parallelise over patches: first by simply using Matlab's \texttt{parfor}.   
Probably not using a \textsc{gpu} as we probably want to leave \textsc{gpu}s for the black box to utilise within each patch.




\subsection{Projective integration}

To take macroscale time steps, invoke several possible projective integration schemes: simple Euler projection, Heun-like method, etc
\citep{Samaey08}.
Need to decide how long a microscale burst needs to be.

Should not need an implicit scheme as the fast dynamics are meant to be only in the micro variables, and the slow dynamics only in the macroscale variables.
However, it could be that the macroscale variables have fast oscillations and it is only the amplitude of the oscillations that are slow.  
Perhaps need to detect and then fix or advise.

A further stage is to implement a projective integration scheme for stochastic macroscale variables: this is important because the averaging over a core of microscale roughness will almost invariably have at least some stochastic legacy effect.
\cite{Calderon2007} did some useful research on stochastic projective intergration.




\subsection{Lift to many internal modes}

In most problems the number of macroscale variables at any given position in space,~\(d_{\Uv}\), is less than the number of microscale variables at a position,~\(d_{\uv}\); often much less \citep[e.g.]{Kevrekidis09a}.
In this case, every time we start a patch simulation we need to provide  \(d_{\uv}-d_{\Uv}\) data at each position in the patch: this is lifting.
The first methodology is to first guess, then run repeated short bursts with reinitialisation, until the simulation reaches a slow manifold.
Then run the real simulation.

If the time taken to reach a local quasi-equilibrium is too long, then it is likely that the macroscale closure is bad and the macroscale variables need to be extended.

A second step is to cater for cases where the slow manifold is stochastic or is surrounded by fast waves: when it is hard to detect the slow manifold, or the slow manifold is not attractive.





\subsection{Macroscale closure}

In some circumstances a researcher\slash practitioner will not code the appropriately set of macroscale variables for a complete closure of the macroscale.
For example, in thin film fluid dynamics at low Reynolds number the only macroscale variable is the fluid depth; however, at higher Reynolds number, circa ten, the inertia of the fluid becomes important and the macroscale variables must additionally include a measure of the mean lateral velocity\slash momentum \citep[e.g.]{Roberts99b}.

At some stage we need to detect any flaw in the closure, and perhaps suggest additional appropriate macroscale variables, or at least their characteristics.
Indeed, a poor closure and a stochastic slow manifold are really two faces of the same problem: the problem is that the chosen macroscale variables do not have a unique evolution in terms of themselves. 
A good resolution of the issue will account for both faces.




\subsection{Exascale fault tolerance}

Matlab is probably not an appropriate vehicle to deal with real exascale faults.  
However, we should cater by coding procedures for fault tolerance and testing them at least synthetically.
Eventually provide hooks to a user routine to be invoked under various potential scenarios.
The nature of fault tolerant algorithms will vary depending upon the scenario, even assuming that each patch burst is executed on one \cpu\ (or closely coupled \cpu{}s): if there are much more \cpu{}s than patches, then maybe simply duplicate all patch simulations;  if much less \cpu{}s than patches, then an asynchronous scheduling of patch bursts should effectively cater for recomputation of failed bursts; if comparable \cpu{}s to patches, then more subtle action is needed.

Once mesotime communication and projective integration is provided, a recomputation approach to intermittent hardware faults should be effective because we then have the tools to restart a burst from available macroscale data.
Should also explore proceeding with a lower order interpolation that misses the faulty burst---because an isolated lower order interpolation probably will not affect the global order of error (it does not in approximating some boundary conditions \citep{Gustafsson1975, Svard2006}




\subsection{Link to established packages}

Several molecular\slash particle\slash agent based codes are well developed and used by a wide community of researchers.  
Plan to develop hooks to use some such codes as the microscale simulators on patches.
First, plan to connect to \textsc{lammps} \cite[]{LAMMPS}.
Second, will evaluate performance, issues, and then consider what other established packages are most promising.

