%Initially by AJR, Apr 2017 -- Nov 2018
%!TEX root = eqnFreeDevMan.tex
\chapter{Introduction}
%\localtableofcontents

\begin{devMan}
This Developers Manual contains complete descriptions of the code in each function in the toolbox, and each example.  For concise descriptions of each function, quick start guides, and some basic examples, see the User Manual.
\end{devMan}


\paragraph{Users}
Place the folder of this toolbox in a path searched by \script.
Then read the section(s) that documents the function of interest.


\paragraph{Quick start}
Maybe start by adapting one of the included examples. Many of the main functions include, at their start, example code of their use (code which is executed if the function is invoked without any arguments).
\begin{itemize}
\item To projectively integrate over time a multiscale, slow-fast, system of \ode{}s you could use \verb|PIRK2()|: adapt the Michaelis--Menten example at the start of \verb|PIRK2.m| (\cref{sec:pirk2eg}).
\item You may use forward bursts of simulation in order to simulate the slow dynamics backward in time, as in \verb|egPIMM.m| (\cref{sec:egPIMM}).
\item To only resolve the slow dynamics in the projective integration, use lifting and restriction functions by adapting the singular perturbation \ode\ example at the start of \verb|PIG.m| (\cref{sec:pigeg}).
\item Consider an evolving system over a large spatial domains when all you have is a microscale code.  To efficiently simulate over the large domain, one can simulate in just small patches of the domain, appropriately coupled:
\begin{itemize}
\item in 1D adapt the code at the start of \verb|configPatches1.m| for Burgers' \pde\ (\cref{sec:configPatches1eg})%
\ifcsname r@sec:wave2D\endcsname, or the staggered patches of 1D water wave equations in \verb|waterWaveExample.m| (\cref{sec:waterWaveExample})\fi;
\item in 2D adapt the code at the start of \verb|configPatches2.m| for nonlinear diffusion (\cref{sec:configPatches2eg})%
\ifcsname r@sec:wave2D\endcsname, or the regular patches of the 2D wave equation of \verb|wave2D.m| (\cref{sec:wave2D})\fi.
\end{itemize}
\item The above are for systems that have \emph{smooth} spatial structures on the microscale: when the microscale is `rough' with a known period (so far only in 1D), then adapt 
the example of \verb|HomogenisationExample.m| (\cref{sec:HomogenisationExample}).
\end{itemize}


\paragraph{Blackbox scenarios} 
Suppose that you have a \emph{detailed and trustworthy} computational simulation of some problem of interest.
Let's say the simulation is coded in terms of detailed (microscale) variable values~\(\uv(t)\), in~\(\RR^{p}\) for any \(p=1,2,\ldots,\infty\), and evolving time~\(t\).
The details~\uv\ could represent particles, agents, states of a system.
When the computation is too time consuming to simulate all the times of interest, then Projective Integration may be able to predict long-time dynamics.  
In this case, provide your detailed computational simulation as a `black box' to the Projective Integration functions of \cref{sec:ProjInt}.

In many scenarios, the problem of interest involves space or a `spatial' lattice.
Let's say that indices~\(i\) correspond to `spatial' coordinates~\(\xv_i(t)\), which are often fixed: in lattice problems the positions~\(\xv_i\) would be fixed in time (unless employing a moving mesh on the microscale); in particle problems the positions would evolve.
And suppose your detailed and trustworthy simulation is coded in terms of micro-field variable values~\(\uv_i(t)\in\RR^p\) at time~\(t\).
Often the detailed computational simulation is too expensive over all the desired spatial domain \(\xv\in\XX\subset\RR^{d}\).
In this case, the toolbox functions of \cref{sec:patch} empower you to simulate on only small, well-separated, patches of space by appropriately coupling between patches your simulation code, as a `black box', executing on each patch. 
The computational savings may be enormous, especially if combined with projective integration.



\paragraph{Contributors}
The aim of this project is to collectively develop a \script\ toolbox of equation-free algorithms.
Initially the algorithms are basic, and the plan is to subsequently develop more and more capability.

\Matlab\ appears a good choice for a first version since it is widespread, efficient, supports various parallel modes, and development costs are reasonably low.
Further it is built on \textsc{blas} and \textsc{lapack} so the cache and superscalar \cpu{} are potentially well utilised.
We aim to develop functions that work for~\script.
\ifcsname r@sec:contribute\endcsname\cref{sec:contribute} outlines some details for contributors.\fi






%\chapter{Overview of major functions and example scripts}
%\label{sec:smf}
%\localtableofcontents
%
%{% excluding the body gives overview of each function/script
%    \renewcommand{\label}[1]{}%
%    \let\section\section%
%    \let\section\subsection%
%    \let\subsection\paragraph%
%    \let\paragraph\subparagraph%
%    \fancyvrbStartStop%
%    \excludeversion{devMan}
%    % input *.m files for Projective Integration  AJR, Oct 2017
%!TEX root = ../Doc/equationFreeDoc.tex
\section{Projective integration of deterministic ODEs}
\label{sec:ProjInt}
\localtableofcontents

This section provides some good projective integration functions \cite[e.g.]{Gear02b, Gear03c, Givon06, Sieber2018}.
The goal is to enable computationally expensive dynamic simulations to be run over long time scales.
\begin{userExample} 
Perhaps start by looking at \cref{sec:egPIMM} which codes the introductory example of a long time simulation of the Michaelis--Menton multiscale system of differential equations.
\end{userExample}

\paragraph{Scenario}
When you are interested in a complex system with many interacting parts or agents, you usually are primarily interested in the self-organised emergent macroscale characteristics.
Projective integration empowers us to efficiently simulate such long-time emergent dynamics.
We suppose you have coded some accurate, fine scale simulation of the complex system, and call such code a microsolver.

The Projective Integration section of this toolbox consists of several functions.
Each function implements over the long-time scale a variant of a standard numerical method to simulate the emergent dynamics of the complex system.
Each function has standardised inputs and outputs.


\paragraph{Main functions}
\begin{itemize}
\item Projective Integration by second or fourth order Runge--Kutta, \verb|PIRK2()| and \verb|PIRK4()| respectively.
These schemes are suitable for precise simulation of the slow dynamics, provided the time period spanned by an application of the microsolver is not too large.
\item Projective Integration with a General solver, \verb|PIG()|.
This function enables a Projective Integration implementation of any solver with macroscale time steps.
It does not matter whether the solver is a standard Matlab or Octave algorithm, or one supplied by the user.
As explored in later examples, \verb|PIG()| should only be used in very stiff systems. 
\end{itemize}
The above functions share dependence on a user-specified `microsolver', that accurately simulates some problem of interest. 
\paragraph{Minor functions}
\begin{itemize}
\item `Constraint-defined manifold computing', \verb|cdmc()|.
This helper function, based on the method introduced in \cite{GearKevrekidis05},  iteratively applies the microsolver and projects the output backwards in time.
The result is to constrain the fast variables close to the slow manifold, without advancing the current time by the duration of an application of the microsolver.
This function can be used to reduce errors related to the simulation length of the microsolver in either the \verb|PIRK| or \verb|PIG| functions.
In particular, it enables \verb|PIG()| to be used on problems that are not particularly stiff.
\item Black box microsolver generator, \verb|bbgen()|.
This simple function takes as input a standard solver with a recommended time step for microscale simulation, and returns a `black box' microsolver for the Projective Integration functions.
\end{itemize}

The following sections describe the \verb|PIRK2()| and \verb|PIG()| functions in detail, providing an example for each.
Then \verb|PIRK4()| is very similar to \verb|PIRK2()|.
Descriptions for the minor functions follow, and an example of the use of \verb|cdmc()|.

\input{../ProjInt/PIRK2.m}
\input{../ProjInt/egPIMM.m}
\input{../ProjInt/PIG.m}
\input{../ProjInt/PIRK4.m}
\begin{body}
\input{../ProjInt/PIRK_Example.m}
\input{../ProjInt/PIG_Example.m}

\subsection{Minor functions}
\label{sec:extras}
\input{../ProjInt/cdmc.m}
\input{../ProjInt/bbgen.m}
\input{../ProjInt/PIG_Explore.m}



\subsection{To do/discuss}
\begin{itemize}
\item could implement Projective Integration by `arbitrary' Runge--Kutta scheme; that is, by having the user input a particular Butcher table---surely only specialists would be interested
\item can `reverse' the order of projection and microsolver applications with a little fiddling.
Then output at each user-requested coarse time is the end point of an application of the microsolver - better predictions for fast variables.
\item Can maybe implement microsolvers that terminate a burst when the fast dynamics have settled using, for example, the 'Events' function handle in ode23. 
\end{itemize}

\end{body}



%    % input *.m files for ... AJR, Nov 2017
%!TEX root = ../Doc/equationFreeDoc.tex
\section{Patch scheme for given microscale discrete space system}
\label{sec:patch}
\localtableofcontents

The patch scheme applies to spatio-temporal systems where
the spatial domain is larger than what can be computed in
reasonable time. Then one may simulate only on small patches
of the space-time domain, and produce correct macroscale
predictions by craftily coupling the patches across
unsimulated space \cite[e.g.]{Hyman2005, Samaey03b,
Samaey04, Roberts06d, Liu2015}.

The spatial structure is to be on a lattice such as obtained
from finite difference approximation of a \pde. Usually
continuous in time.

\paragraph{Quick start}
For an example, see \cref{sec:configPatches1eg,sec:configPatches2eg}  for basic code that uses the provided functions to simulate Burgers'~\pde\ and a nonlinear `diffusion' \pde.

\input{../Patch/configPatches1.m}
\input{../Patch/patchSmooth1.m}
\input{../Patch/patchEdgeInt1.m}
\begin{userExample}
\input{../Patch/BurgersExample.m}
\input{../Patch/HomogenisationExample.m}
%\input{../Patch/configPatches1.m}
%\input{../Patch/patchCoreSmooth1.m}
%\input{../Patch/patchCoreEdgeInt1.m}
%\input{../Patch/BurgersExample.m}
%\input{../Patch/HomogenisationCoreExample.m}
\input{../Patch/waterWaveExample.m}
\end{userExample}
% 2D stuff
\input{../Patch/configPatches2.m}
\input{../Patch/patchSmooth2.m}
\input{../Patch/patchEdgeInt2.m}
\begin{userExample}
\input{../Patch/wave2D.m}



\subsection{To do}
\begin{itemize}
\item Testing is so far only qualitative.  Need to be quantitative.
\item Multiple space dimensions.
\item Heterogeneous microscale via averaging regions.
\item Parallel processing versions.
\item ??
\item Adapt to maps in micro-time?  Surely easy, just an example.
\end{itemize}


\subsection{Miscellaneous tests}
\input{../Patch/patchEdgeInt1test.m}
\input{../Patch/patchEdgeInt2test.m}

\end{userExample}

%    \includeversion{devMan}
%}%end-exclusion

