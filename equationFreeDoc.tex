\documentclass[11pt,a5paper,twoside]{article}
\IfFileExists{ajr.sty}{\usepackage{ajr}}{
  \IfFileExists{jm.sty}{\usepackage{jm}}{
    \usepackage[a5paper,margin=11mm]{geometry}}}

\title{Equation-Free function toolbox for Matlab/Octave}

\author{A. J. Roberts\thanks{%
\url{http://www.maths.adelaide.edu.au/anthony.roberts},
\url{http://orcid.org/0000-0001-8930-1552}}
\and et al.\thanks{Be the first to appear here for your contribution.}}

\date{\today}


% uncomment the sections to typeset
\includeonly{
introduction
,ProjInt/projInt
%%%%,ProjIntDMD/projIntDMD
,Patch/patch
%%%%,RKInt/rk2int
%,apptpd
}

% verbatim listing is quick and reliable
%\usepackage{verbatim}
%\newenvironment{matlab}{\verbatim}{\endverbatim}
% But fancyvrb does more, such as line numbers
\usepackage{fancyvrb}
\newenvironment{matlab}%
    {\Verbatim[numbers=left,firstnumber=\the\inputlineno]}%
    {\endVerbatim}
% also get fancyvrb to omit %{ and %} pairs, but requires they always be used
\makeatletter
\def\fancyvrbStartStop{%
  \edef\FancyVerbStartString{\@percentchar\@charrb} 
  \edef\FancyVerbStopString{\@percentchar\@charlb} }
\makeatother
% could change appearance with e.g. \renewcommand{\theFancyVerbLine}{%
%  \textcolor{red}{\small 8.\alph{FancyVerbLine}}}

%\usepackage{mcode} %[framed,numbered,autolinebreaks,useliterate]
%\renewenvironment{matlab}{\lstlisting}{\endlstlisting}

%\usepackage{pgfplots}\pgfplotsset{compat=newest}% avoid as too few use??


\usepackage{natbib}
\bibliographystyle{agsm}
\let\harvardurl\url

\usepackage{etocx}% for mini-local-tableofconents
\usepackage{microtype,amsmath,defns,graphicx,hyperref}
% Invoke cleverref
\usepackage[capitalise,nameinlink,noabbrev]{cleveref}
\crefname{equation}{}{}
% Default "Item" useless, use enumitem and ref=
\crefname{enumi}{}{}
\crefname{enumii}{}{}
\crefname{enumiii}{}{}
\crefname{enumiv}{}{}

% command definitions
\def\script{\textsc{Matlab}\slash Octave}
\newcommand{\cpu}{\textsc{cpu}}
\Vec f\Vec x\Vec u \Vec X \Vec U
\Bb R  \Bb X



% These are recommended by Rob J Hyndman (2011)
% \footnote{\url{http://robjhyndman.com/researchtips/latex-floats/}}
\setcounter{topnumber}{2}
\setcounter{bottomnumber}{2}
\setcounter{totalnumber}{4}
\renewcommand{\topfraction}{0.85}
\renewcommand{\bottomfraction}{0.85}
\renewcommand{\textfraction}{0.15}
\renewcommand{\floatpagefraction}{0.7}




\begin{document}
\epstopdfsetup{suffix=} % must be after begin{document}

\maketitle

\begin{abstract}
This `equation-free toolbox' facilitates the computer-assisted analysis of complex, multiscale systems.
Its aim is to enable microscopic simulators to perform system level tasks and analysis.
The methodology bypasses the derivation of macroscopic evolution equations by using only short bursts of microscale simulations which are often the best available description of a system
\cite[e.g.]{Kevrekidis09a, Kevrekidis04a, Kevrekidis03b}.
This suite of functions should empower users to start implementing such methods---but so far we have only just started.
\end{abstract}

\tableofcontents

%Initially by AJR, Apr 2017 -- Apr 2019
%!TEX root = eqnFreeDevMan.tex
\chapter{Introduction}
%\localtableofcontents

\begin{devMan}
This Developers Manual contains complete descriptions of the code in each function in the toolbox, and of each example.  For concise descriptions of each function, quick start guides, and some basic examples, see the User Manual.
\end{devMan}


\paragraph{Users}
Download via \url{https://github.com/uoa1184615/EquationFreeGit}.
Place the folder of this toolbox in a path searched by \script.
Then read the section(s) that documents the function of interest.


\paragraph{Quick start}
Maybe start by adapting one of the included examples. 
Many of the main functions include, at their beginning, example code of their use---code which is executed when the function is invoked without any arguments.
\begin{itemize}
\item To projectively integrate over time a multiscale, slow-fast, system of \ode{}s you could use \verb|PIRK2()|, or \verb|PIRK4()| for higher-order accuracy: adapt the Michaelis--Menten example at the beginning of \verb|PIRK2.m| (\cref{sec:pirk2eg}).
\item You may use forward bursts of simulation in order to simulate the slow dynamics backward in time, as in \verb|egPIMM.m| (\cref{sec:egPIMM}).
\item To only resolve the slow dynamics in the projective integration, use lifting and restriction functions by adapting the singular perturbation \ode\ example at the beginning of \verb|PIG.m| (\cref{sec:pigeg}).

\paragraph{Space-time systems}
Consider an evolving system over a large spatial domains when all you have is a microscale code.  
To efficiently simulate over the large domain, one can simulate in just small patches of the domain, appropriately coupled.
\item In 1D adapt the code at the beginning of \verb|configPatches1.m| for Burgers' \pde\ (\cref{sec:configPatches1eg})%
\ifcsname r@sec:wave2D\endcsname, or the staggered patches of 1D water wave equations in \verb|waterWaveExample.m| (\cref{sec:waterWaveExample})\fi.
\item in 2D adapt the code at the beginning of \verb|configPatches2.m| for nonlinear diffusion (\cref{sec:configPatches2eg})%
\ifcsname r@sec:wave2D\endcsname, or the regular patches of the 2D wave equation of \verb|wave2D.m| (\cref{sec:wave2D})\fi.

\item The above two are for systems that have \emph{smooth} spatial structures on the microscale: when the microscale is `rough' with a known period (so far only in 1D), then adapt 
the example of \verb|HomogenisationExample.m| (\cref{sec:HomogenisationExample}).
\end{itemize}


\paragraph{Verification}
Most of these schemes have proven `accuracy' when compared to the underlying specified microscale system.
In the spatial patch schemes, we measure `accuracy' by the order of consistency between macroscale dynamics and the specified microscale.  
\begin{itemize}
\item \cite{Roberts06d} and \cite{Roberts2011a} proved reasonably general high-order consistency for the 1D and 2D patch schemes, respectively.
\item In wave-like systems, \cite{Cao2014a} established high-order consistency for the 1D staggered patch scheme.
\item A heterogeneous microscale is more difficult, but \cite{Bunder2013b} showed good accuracy in a variety of circumstances, for appropriately chosen parameters. 
\end{itemize}



\paragraph{Blackbox scenarios} 
Suppose that you have a \emph{detailed and trustworthy} computational simulation of some problem of interest.
Let's say the simulation is coded in terms of detailed (microscale) variable values~\(\uv(t)\), in~\(\RR^{p}\) for some~\(p\), and evolving time~\(t\).
The details~\uv\ could represent particles, agents, or states of a system.
When the computation is too time consuming to simulate all the times of interest, then Projective Integration may be able to predict long-time dynamics, both forward and backward in time.  
In this case, provide your detailed computational simulation as a `black box' to the Projective Integration functions of \cref{sec:ProjInt}.

In many scenarios, the problem of interest involves space or a `spatial' lattice.
Let's say that indices~\(i\) correspond to `spatial' coordinates~\(\xv_i(t)\), which are often fixed: in lattice problems the positions~\(\xv_i\) would be fixed in time (unless employing a moving mesh on the microscale); however, in particle problems the positions would evolve.
And suppose your detailed and trustworthy simulation is coded also in terms of micro-field variable values~\(\uv_i(t)\in\RR^p\) at time~\(t\).
Often the detailed computational simulation is too expensive over all the desired spatial domain \(\xv\in\XX\subset\RR^{d}\).
In this case, the toolbox functions of \cref{sec:patch} empower you to simulate on only small, well-separated, patches of space by appropriately coupling between patches your simulation code, as a `black box', executing on each small patch. 
The computational savings may be enormous, especially if combined with projective integration.



\paragraph{Contributors}
The aim of this project is to collectively develop a \script\ toolbox of equation-free algorithms.
Initially the algorithms are basic, and the plan is to subsequently develop more and more capability.

\Matlab\ appears a good choice for a first version since it is widespread, efficient, supports various parallel modes, and development costs are reasonably low.
Further it is built on \textsc{blas} and \textsc{lapack} so the cache and superscalar \cpu{} are potentially well utilised.
We aim to develop functions that work for~\script.
\ifcsname r@sec:contribute\endcsname\cref{sec:contribute} outlines some details for contributors.\fi





\fancyvrbStartStop

% input *.m files for Projective Integration  AJR, Oct 2017
%!TEX root = ../Doc/equationFreeDoc.tex
\section{Projective integration of deterministic ODEs}
\label{sec:ProjInt}
\localtableofcontents

This section provides some good projective integration functions \cite[e.g.]{Gear02b, Gear03c, Givon06, Sieber2018}.
The goal is to enable computationally expensive dynamic simulations to be run over long time scales.
\begin{userExample} 
Perhaps start by looking at \cref{sec:egPIMM} which codes the introductory example of a long time simulation of the Michaelis--Menton multiscale system of differential equations.
\end{userExample}

\paragraph{Scenario}
When you are interested in a complex system with many interacting parts or agents, you usually are primarily interested in the self-organised emergent macroscale characteristics.
Projective integration empowers us to efficiently simulate such long-time emergent dynamics.
We suppose you have coded some accurate, fine scale simulation of the complex system, and call such code a microsolver.

The Projective Integration section of this toolbox consists of several functions.
Each function implements over the long-time scale a variant of a standard numerical method to simulate the emergent dynamics of the complex system.
Each function has standardised inputs and outputs.


\paragraph{Main functions}
\begin{itemize}
\item Projective Integration by second or fourth order Runge--Kutta, \verb|PIRK2()| and \verb|PIRK4()| respectively.
These schemes are suitable for precise simulation of the slow dynamics, provided the time period spanned by an application of the microsolver is not too large.
\item Projective Integration with a General solver, \verb|PIG()|.
This function enables a Projective Integration implementation of any solver with macroscale time steps.
It does not matter whether the solver is a standard Matlab or Octave algorithm, or one supplied by the user.
As explored in later examples, \verb|PIG()| should only be used in very stiff systems. 
\end{itemize}
The above functions share dependence on a user-specified `microsolver', that accurately simulates some problem of interest. 
\paragraph{Minor functions}
\begin{itemize}
\item `Constraint-defined manifold computing', \verb|cdmc()|.
This helper function, based on the method introduced in \cite{GearKevrekidis05},  iteratively applies the microsolver and projects the output backwards in time.
The result is to constrain the fast variables close to the slow manifold, without advancing the current time by the duration of an application of the microsolver.
This function can be used to reduce errors related to the simulation length of the microsolver in either the \verb|PIRK| or \verb|PIG| functions.
In particular, it enables \verb|PIG()| to be used on problems that are not particularly stiff.
\item Black box microsolver generator, \verb|bbgen()|.
This simple function takes as input a standard solver with a recommended time step for microscale simulation, and returns a `black box' microsolver for the Projective Integration functions.
\end{itemize}

The following sections describe the \verb|PIRK2()| and \verb|PIG()| functions in detail, providing an example for each.
Then \verb|PIRK4()| is very similar to \verb|PIRK2()|.
Descriptions for the minor functions follow, and an example of the use of \verb|cdmc()|.

\input{../ProjInt/PIRK2.m}
\input{../ProjInt/egPIMM.m}
\input{../ProjInt/PIG.m}
\input{../ProjInt/PIRK4.m}
\begin{body}
\input{../ProjInt/PIRK_Example.m}
\input{../ProjInt/PIG_Example.m}

\subsection{Minor functions}
\label{sec:extras}
\input{../ProjInt/cdmc.m}
\input{../ProjInt/bbgen.m}
\input{../ProjInt/PIG_Explore.m}



\subsection{To do/discuss}
\begin{itemize}
\item could implement Projective Integration by `arbitrary' Runge--Kutta scheme; that is, by having the user input a particular Butcher table---surely only specialists would be interested
\item can `reverse' the order of projection and microsolver applications with a little fiddling.
Then output at each user-requested coarse time is the end point of an application of the microsolver - better predictions for fast variables.
\item Can maybe implement microsolvers that terminate a burst when the fast dynamics have settled using, for example, the 'Events' function handle in ode23. 
\end{itemize}

\end{body}




%% input *.m files for Projective Integration  AJR, Oct 2017
%!TEX root = ../equationFreeDoc.tex
\section{Projective integration of deterministic ODEs via DMD}
\label{sec:ProjIntDMD}
\localtableofcontents

This is a very first stab at a good projective integration function that uses DMD.

% input *.m files for Projective Integration  AJR, Oct 2017
%!TEX root = ../equationFreeDoc.tex
\section{Projective integration of deterministic ODEs via DMD}
\label{sec:ProjIntDMD}
\localtableofcontents

This is a very first stab at a good projective integration function that uses DMD.

% input *.m files for Projective Integration  AJR, Oct 2017
%!TEX root = ../equationFreeDoc.tex
\section{Projective integration of deterministic ODEs via DMD}
\label{sec:ProjIntDMD}
\localtableofcontents

This is a very first stab at a good projective integration function that uses DMD.

\input{ProjIntDMD/projIntDMD.m}
\input{ProjIntDMD/projIntDMDExample1.m}
\input{ProjIntDMD/projIntDMDPatches.m}
\input{ProjIntDMD/projIntDMDExplore1.m}
\input{ProjIntDMD/projIntDMDExplore2.m}

\subsection{To do}
\begin{itemize}
\item Check the order of accuracy of the algorithm.
\item Develop theory quantitively justifying the \dmd\ approach.
\item Develop techniques to automatically make some of the decisions about step-sizes, burst lengths, rank, and so on.
\item Develop higher accuracy versions (once we have some idea about current accuracy).
\item Adapt approach to algorithms for stochastic systems.
\end{itemize}


\input{ProjIntDMD/projIntDMDExample1.m}
\input{ProjIntDMD/projIntDMDPatches.m}
\input{ProjIntDMD/projIntDMDExplore1.m}
\input{ProjIntDMD/projIntDMDExplore2.m}

\subsection{To do}
\begin{itemize}
\item Check the order of accuracy of the algorithm.
\item Develop theory quantitively justifying the \dmd\ approach.
\item Develop techniques to automatically make some of the decisions about step-sizes, burst lengths, rank, and so on.
\item Develop higher accuracy versions (once we have some idea about current accuracy).
\item Adapt approach to algorithms for stochastic systems.
\end{itemize}


\input{ProjIntDMD/projIntDMDExample1.m}
\input{ProjIntDMD/projIntDMDPatches.m}
\input{ProjIntDMD/projIntDMDExplore1.m}
\input{ProjIntDMD/projIntDMDExplore2.m}

\subsection{To do}
\begin{itemize}
\item Check the order of accuracy of the algorithm.
\item Develop theory quantitively justifying the \dmd\ approach.
\item Develop techniques to automatically make some of the decisions about step-sizes, burst lengths, rank, and so on.
\item Develop higher accuracy versions (once we have some idea about current accuracy).
\item Adapt approach to algorithms for stochastic systems.
\end{itemize}



% input *.m files for ... AJR, Nov 2017
%!TEX root = ../Doc/equationFreeDoc.tex
\section{Patch scheme for given microscale discrete space system}
\label{sec:patch}
\localtableofcontents

The patch scheme applies to spatio-temporal systems where
the spatial domain is larger than what can be computed in
reasonable time. Then one may simulate only on small patches
of the space-time domain, and produce correct macroscale
predictions by craftily coupling the patches across
unsimulated space \cite[e.g.]{Hyman2005, Samaey03b,
Samaey04, Roberts06d, Liu2015}.

The spatial structure is to be on a lattice such as obtained
from finite difference approximation of a \pde. Usually
continuous in time.

\paragraph{Quick start}
For an example, see \cref{sec:configPatches1eg,sec:configPatches2eg}  for basic code that uses the provided functions to simulate Burgers'~\pde\ and a nonlinear `diffusion' \pde.

\input{../Patch/configPatches1.m}
\input{../Patch/patchSmooth1.m}
\input{../Patch/patchEdgeInt1.m}
\begin{userExample}
\input{../Patch/BurgersExample.m}
\input{../Patch/HomogenisationExample.m}
%\input{../Patch/configPatches1.m}
%\input{../Patch/patchCoreSmooth1.m}
%\input{../Patch/patchCoreEdgeInt1.m}
%\input{../Patch/BurgersExample.m}
%\input{../Patch/HomogenisationCoreExample.m}
\input{../Patch/waterWaveExample.m}
\end{userExample}
% 2D stuff
\input{../Patch/configPatches2.m}
\input{../Patch/patchSmooth2.m}
\input{../Patch/patchEdgeInt2.m}
\begin{userExample}
\input{../Patch/wave2D.m}



\subsection{To do}
\begin{itemize}
\item Testing is so far only qualitative.  Need to be quantitative.
\item Multiple space dimensions.
\item Heterogeneous microscale via averaging regions.
\item Parallel processing versions.
\item ??
\item Adapt to maps in micro-time?  Surely easy, just an example.
\end{itemize}


\subsection{Miscellaneous tests}
\input{../Patch/patchEdgeInt1test.m}
\input{../Patch/patchEdgeInt2test.m}

\end{userExample}


%% input *.m files for Runge--Kutta 2 integration  AJR, Oct 2017
%!TEX root = ../Doc/equationFreeDoc.tex
\section{Runge--Kutta 2 integration of deterministic ODEs}
\label{sec:rk2int}
\secttoc

This describes a simple RK2 integration function, and some tests, as an example of the layout of information.

% input *.m files for Runge--Kutta 2 integration  AJR, Oct 2017
%!TEX root = ../Doc/equationFreeDoc.tex
\section{Runge--Kutta 2 integration of deterministic ODEs}
\label{sec:rk2int}
\secttoc

This describes a simple RK2 integration function, and some tests, as an example of the layout of information.

% input *.m files for Runge--Kutta 2 integration  AJR, Oct 2017
%!TEX root = ../Doc/equationFreeDoc.tex
\section{Runge--Kutta 2 integration of deterministic ODEs}
\label{sec:rk2int}
\secttoc

This describes a simple RK2 integration function, and some tests, as an example of the layout of information.

\input{../RKInt/rk2int.m}
\input{../RKInt/rk2intTest1.m}
\input{../RKInt/rk2intTest2.m}

\input{../RKInt/rk2intTest1.m}
\input{../RKInt/rk2intTest2.m}

\input{../RKInt/rk2intTest1.m}
\input{../RKInt/rk2intTest2.m}


\appendix
%Initiated by AJR, Apr 2017
%!TEX root = equationFreeDoc.tex
\section{Aspects of developing a `toolbox' for patch dynamics}

This appendix documents sketchy further thoughts on aspects of the development.


\subsection{Macroscale grid}

The patches are to be distributed on a macroscale grid: the \(j\)th~patch `centred' at position~\(\Xv_j\in\XX\).
In principle the patches could move, but let's keep them fixed in the first version.
The simplest macroscale grid will be rectangular (\texttt{meshgrid}), but we plan to allow a deformed grid to secondly cater for boundary fitting to quite general domain shapes~\XX.
And plan to later allow for more general interconnect networks for more topologies in application.




\subsection{Macroscale field variables}

The researcher\slash practitioner has to know an appropriate set of macroscale field variables~\(\Uv(t)\in\RR^{d_{\Uv}}\) for each patch.  
For example, first they might be a simple average over a core of a patch of all of the micro-field variables; second, they might be a subset of the average micro-field variables; and third in general the macro-variables might be a nonlinear function of the micro-field variables (such as temperature is the average speed squared).
The core might be just one point, or a sizeable fraction of the patch.

The mapping from microscale variable to macroscale variables is often termed the restriction.

In practice, users may not choose an appropriate set of macro-variables, so will eventually need to code some diagnostic to indicate a failure of the assumed closure.




\subsection{Boundary and coupling conditions}

The physical domain boundary conditions are distinct from the conditions coupling the patches together.
Start with physical boundary conditions of periodicity in the macroscale.

Second, assume the physical boundary conditions are that the macro-variables are known at macroscale grid points around the boundary.  
Then the issue is to adjust the interpolation to cater for the boundary presence and shape.
The coupling conditions for the patches should cater for the range of Robin-like boundary conditions, from Dirichlet to Neumann.
Two possibilities arise: direct imposition of the coupling action \citep{Roberts06d}, or control by the action.

Third, assume that some of the patches have some edges coincident with the boundary of the macroscale domain~\XX, and it is on these edges that macroscale physical boundary conditions are applied.
Then the interpolation from the core of these edge patches is the same as the second case of prescribed boundary macro-variables.
An issue is that each boundary patch should be big enough to cater for any spatial boundary layers transitioning from the applied boundary condition to the interior slow evolution.

Alternatively, we might have the physical boundary condition constrain the interpolation between patches.

Often microscale simulations are easiest to write when `periodic' in microscale space.  
To cater for this we should also allow a control at perhaps the quartiles of a micro-periodic simulator.





\subsection{Mesotime communication}

Since communication limits large scale parallelism, a first step in reducing communication will be to implement only updating the coupling conditions when necessary.
Error analysis indicates that updating on times longer the microscale times and shorter than the macroscale times can be effective
\citep{Bunder2015a}.
Implementations can communicate one or more derivatives in time, as well as macroscale variables.

At this stage we can effectively parallelise over patches: first by simply using Matlab's \texttt{parfor}.   
Probably not using a \textsc{gpu} as we probably want to leave \textsc{gpu}s for the black box to utilise within each patch.




\subsection{Projective integration}

To take macroscale time steps, invoke several possible projective integration schemes: simple Euler projection, Heun-like method, etc
\citep{Samaey08}.
Need to decide how long a microscale burst needs to be.

Should not need an implicit scheme as the fast dynamics are meant to be only in the micro variables, and the slow dynamics only in the macroscale variables.
However, it could be that the macroscale variables have fast oscillations and it is only the amplitude of the oscillations that are slow.  
Perhaps need to detect and then fix or advise.

A further stage is to implement a projective integration scheme for stochastic macroscale variables: this is important because the averaging over a core of microscale roughness will almost invariably have at least some stochastic legacy effect.




\subsection{Lift to many internal modes}

In most problems the number of macroscale variables at any given position in space,~\(d_{\Uv}\), is less than the number of microscale variables at a position,~\(d_{\uv}\); often much less \citep[e.g.]{Kevrekidis09a}.
In this case, every time we start a patch simulation we need to provide  \(d_{\uv}-d_{\Uv}\) data at each position in the patch: this is lifting.
The first methodology is to first guess, then run repeated short bursts with reinitialisation, until the simulation reaches a slow manifold.
Then run the real simulation.

If the time taken to reach a local quasi-equilibrium is too long, then it is likely that the macroscale closure is bad and the macroscale variables need to be extended.

A second step is to cater for cases where the slow manifold is stochastic or is surrounded by fast waves: when it is hard to detect the slow manifold, or the slow manifold is not attractive.





\subsection{Macroscale closure}

In some circumstances a researcher\slash practitioner will not code the appropriately set of macroscale variables for a complete closure of the macroscale.
For example, in thin film fluid dynamics at low Reynolds number the only macroscale variable is the fluid depth; however, at higher Reynolds number, circa ten, the inertia of the fluid becomes important and the macroscale variables must additionally include a measure of the mean lateral velocity\slash momentum \citep[e.g.]{Roberts99b}.

At some stage we need to detect any flaw in the closure, and perhaps suggest additional appropriate macroscale variables, or at least their characteristics.
Indeed, a poor closure and a stochastic slow manifold are really two faces of the same problem: the problem is that the chosen macroscale variables do not have a unique evolution in terms of themselves. 
A good resolution of the issue will account for both faces.




\subsection{Exascale fault tolerance}

Matlab is probably not an appropriate vehicle to deal with real exascale faults.  
However, we should cater by coding procedures for fault tolerance and testing them at least synthetically.
Eventually provide hooks to a user routine to be invoked under various potential scenarios.
The nature of fault tolerant algorithms will vary depending upon the scenario, even assuming that each patch burst is executed on one \cpu\ (or closely coupled \cpu{}s): if there are much more \cpu{}s than patches, then maybe simply duplicate all patch simulations;  if much less \cpu{}s than patches, then an asynchronous scheduling of patch bursts should effectively cater for recomputation of failed bursts; if comparable \cpu{}s to patches, then more subtle action is needed.

Once mesotime communication and projective integration is provided, a recomputation approach to intermittent hardware faults should be effective because we then have the tools to restart a burst from available macroscale data.
Should also explore proceeding with a lower order interpolation that misses the faulty burst---because an isolated lower order interpolation probably will not affect the global order of error (it does not in approximating some boundary conditions \citep{Gustafsson1975, Svard2006}




\subsection{Link to established packages}

Several molecular\slash particle\slash agent based codes are well developed and used by a wide community of researchers.  
Plan to develop hooks to use some such codes as the microscale simulators on patches.
First, plan to connect to \textsc{lammps} \cite[]{LAMMPS}.
Second, will evaluate performance, issues, and then consider what other established packages are most promising.




% bib data from local; merge with 'bibexport equationFreeDoc'
\IfFileExists{ajr.sty}{\bibliography{bibexport,ajr,bib}}{}
\IfFileExists{jm.sty}{\bibliography{bibexport,jm,bib}}{}

\end{document}