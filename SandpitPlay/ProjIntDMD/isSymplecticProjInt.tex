\documentclass[11pt,a5paper]{article}
\IfFileExists{ajr.sty}{\usepackage{ajr}}{}
\usepackage{amsmath,amsfonts,eulervm}

\title{Could there be Symplectic Projective Integration?}
\author{A. J. Roberts\thanks{%
\url{http://www.maths.adelaide.edu.au/anthony.roberts},
\url{http://orcid.org/0000-0001-8930-1552}}}
\date{\today}

\begin{document}

\maketitle

Two symplectic-like scenarios occur to me.  \begin{itemize}
\item The first is when the macro-scale dynamics are nearly those of a `slow' wave system (Hamiltonian etc), but the micro-scale has enough dissipation to damp fast waves on a micro-scale time.
For example, patch simulations of shallow water waves where the micro-scale in patches feels `turbulent eddy' viscosity.
\item The second is where all modes are (near) wave-like, but we want to effectively `average' over the fast waves to find the slow waves.
Maybe \textsc{dmd} would be good here in that it might better detect and `average'-out the fast waves.
\end{itemize}

We may need to assume a user can flag a set of `position' variables~\(p\), and a complementary set of `momentum' variables~\(q\).

The most basic problem and the basic scheme (symplectic Euler) is
\begin{align*}
&\dot p=-aq\,, &&p_{k+1}=p_k+h(-aq_k),
\\&\dot q=+bp\,, &&q_{k+1}=q_k+hbp_{k+1}\,,
\end{align*}
for time-step~\(h\).
\begin{itemize}
\item This differential problem has solutions with frequency~\(\sqrt{ab}\).
Over a time-step~\(h\) the solutions thus have characteristic multiplier of 
\begin{equation*}
\lambda_{\text{exact}}=e^{\pm i\sqrt{ab}h}
=1-\tfrac12 abh^2\pm i\sqrt{ab}h(1-\tfrac16abh^2)+O(h^4).
\end{equation*}

\item The symplectic Euler scheme is semi-implicit:
\begin{equation*}
\begin{bmatrix} 1&0\\-bh&1 \end{bmatrix}
\begin{bmatrix} p_{k+1}\\q_{k+1} \end{bmatrix}
=\begin{bmatrix} 1&-ah\\0&1 \end{bmatrix}
\begin{bmatrix} p_{k}\\q_{k} \end{bmatrix}.
\end{equation*}
Its characteristic multipliers~\(\lambda\) then satisfy
\begin{align*}
\begin{vmatrix} \lambda-1&ah\\-bh\lambda&\lambda-1 \end{vmatrix}
=\lambda^2-(2-abh^2)\lambda+1=0\,,
\end{align*}
with solution
\begin{equation*}
\lambda=1-\tfrac12abh^2\pm i\sqrt{ab}h\sqrt{1-\tfrac14abh^2}
=\lambda_{\text{exact}}+O(h^3).
\end{equation*}
So the scheme is globally~\(O(h^2)\) for such a system.

Further, when \(ab\in\mathbb R^+\) and time-step \(h<\tfrac12\sqrt{ab}\), then the magnitude of the multiplier is beautifully one:
\begin{equation*}
|\lambda|^2=1-abh^2+\tfrac14(ab)^2h^4 +abh^2(1-\tfrac14abh^2)=1\,.
\end{equation*}

\end{itemize}

\paragraph{More generally,}  symplectic Euler methods for \(\dot p=f(p,q)\) and \(\dot q=g(p,q)\) are
\begin{equation*}
\begin{cases}
p_{k+1}=p_k+hf(p_{k+1},q_k)
\\q_{k+1}=q_k+hg(p_{k+1},q_k)
\end{cases}
\quad\text{or}\quad
\begin{cases}
p_{k+1}=p_k+hf(p_{k},q_{k+1})
\\q_{k+1}=q_k+hg(p_{k},q_{k+1})
\end{cases}
\end{equation*}
The first method applied to \(\dot p=\lambda p\) and \(\dot q=\mu q\) gives a mix of implicit and explicit: \(p_{k+1}=\frac1{1-h\lambda}p_k\) and \(q_{k+1}=(1+h\mu)q_k\); and complementary formula for the second method. 
These are locally~\(O(h^2)\), so globally~\(O(h)\).


\paragraph{Challenge}
\emph{Is there a projective integration version of this basic scheme?  
that also works for weakly dissipative dynamics?  
when there is not a clean separation between `position' and `momentum'?}

\end{document}
