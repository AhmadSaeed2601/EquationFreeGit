\documentclass[11pt,a4paper,twoside]{article}

\setcounter{topnumber}{2}
\setcounter{bottomnumber}{2}
\setcounter{totalnumber}{4}
\renewcommand{\topfraction}{0.85}
\renewcommand{\bottomfraction}{0.85}
\renewcommand{\textfraction}{0.15}
\renewcommand{\floatpagefraction}{0.7}


\usepackage{pgfplots} 
\pgfplotsset{compat=1.15} %,tikz
%\IfFileExists{ajr.sty}{}{
%    \pgfplotsset{compat=1.10}
 %   }
\usepgfplotslibrary{external} %exports an external pdf for each tikz figure
\tikzexternalize
% The following forces redraw of all tikz graphics;
\tikzset{external/force remake}
\usetikzlibrary{decorations.markings}
\usetikzlibrary{shapes,arrows}
\usetikzlibrary{positioning}

\begin{document}

\begin{figure}
\centering
\begin{tikzpicture}[node distance = 0.5cm, auto]
\tikzstyle{bigblock} = [rectangle, draw, thick,   text badly centered,fill=red!10, 
    text width=6cm, rounded corners, minimum height=4ex]
\tikzstyle{block} = [rectangle, draw=red!80!black, thick, anchor=west, fill=white,
    text width=5cm, rounded corners, minimum height=4ex]
 \tikzstyle{smallblock} = [rectangle, draw=red!80!black, thick, anchor=west, fill=white,
    text width=3.4cm, rounded corners, minimum height=8ex]   
\tikzstyle{line} = [draw, -latex']
\tikzstyle{lined} = [draw, latex'-latex']
\node [bigblock,draw=red!80!black] (pi) {Projective integration scheme\\
    \begin{tikzpicture}[node distance = 3ex, auto]
    \node [block] (configPatches) {First step};
    \node [block, below=of configPatches] (microPDE) {The next step};
    \node [block, below=of microPDE] (process) {process results and plot};    
    \path [line] (configPatches) -- (microPDE);
    \path [line] (microPDE) -- (process);
    \end{tikzpicture}
    };   
\end{tikzpicture}
\caption{A schematic of the projective integration scheme.}\label{fig:pischeme}
\end{figure}
    
\begin{figure}
\centering
\begin{tikzpicture}[node distance = 3ex, auto]
\tikzstyle{bigblock} = [rectangle, draw, thick,   text badly centered, 
    text width=12cm, rounded corners, minimum height=4ex]
\tikzstyle{block} = [rectangle, draw=blue!80!black, thick, anchor=west, fill=white,
    text width=5cm, rounded corners, minimum height=8ex]
 \tikzstyle{smallblock} = [rectangle, draw=blue!80!black, thick, anchor=west, fill=white,
    text width=3.4cm, rounded corners, minimum height=8ex]   
\tikzstyle{line} = [draw, -latex']
\tikzstyle{lined} = [draw, latex'-latex']
\node [bigblock,draw=blue!80!black,fill=blue!10] (gaptooth) {Gap-tooth scheme\\
    \begin{tikzpicture}[node distance = 3ex, auto]
    \node [block] (configPatches) {\texttt{configPatches1}\\ construct patches across space};
    \node [block, below=of configPatches] (microPDE) {standard \textsc{pde} solver \texttt{ode15s}\\ input  \texttt{patchSmooth1}};
    \node [smallblock, above right=-1.2cm and 0.8cm of microPDE] (patchSmooth1) {\texttt{patchSmooth1}\\ interface with microscale \textsc{pde} and patch coupling conditions};
    \node [smallblock, below left=0.5cm and -1.5cm of patchSmooth1] (coupling) {\texttt{patchEdge1}\\ coupling conditions};
    \node [smallblock, below right=0.5cm and -1.5cm of patchSmooth1] (micropde) {\texttt{heteroDiff}\\ microscale \textsc{pde}};
    \node [block, below=2cm of microPDE] (process) {process results and plot};    
    \path [line] (configPatches) -- (microPDE);
    \path [lined] (microPDE) to[out=0,in=180] (patchSmooth1);
    \path [lined] (patchSmooth1) to[out=240,in=60] (coupling);
    \path [lined] (patchSmooth1) to[out=300,in=120] (micropde);
    \path [line] (microPDE) -- (process);
    \end{tikzpicture}
    };   
\node [bigblock,draw=red!80!black,fill=red!10,below=of gaptooth] (pi) {Projective integration scheme};
 \path [lined] (gaptooth) -- (pi);
\end{tikzpicture}
\caption{A schematic for \texttt{HomogenisationExample}.
This example has two major steps: the gap-tooth scheme (blue) which evaluates the microscale \textsc{pde} within discrete patches; and projective integration (red) for which Figure~\ref{fig:pischeme} provides further details. }\label{fig:homogenisationex}
\end{figure}

\end{document}