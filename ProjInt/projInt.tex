% input *.m files for Projective Integration  AJR, Oct 2017
%!TEX root = ../Doc/eqnFreeDevMan.tex
\chapter{Projective integration of deterministic ODEs}
\label{sec:ProjInt}
\localtableofcontents


\section{Introduction}

This section provides some good projective integration functions \cite[e.g.]{Gear02b, Gear03c, Givon06, Maclean2015, Sieber2018}.
The goal is to enable computationally expensive multiscale dynamic simulations\slash integrations to efficiently compute over very long time~scales.

\paragraph{Quick start} 
\cref{sec:pirk2eg} shows the most basic use of a projective integration function.
\cref{sec:egPIMM} shows how to code more variations of the introductory example of a long time simulation of the Michaelis--Menton multiscale system of differential equations.
Then see \cref{fig:constructPI,fig:PIchoosemacro}

\paragraph{Scenario}
When you are interested in a complex system with many interacting parts or agents, you usually are primarily interested in the self-organised emergent macroscale characteristics.
Projective integration empowers us to efficiently simulate such long-time emergent dynamics.
We suppose you have coded some accurate, fine-scale, microscale simulation of the complex system, and call such code a microsolver.

The Projective Integration section of this toolbox consists of several functions.
Each function implements over a long-time scale a variant of a standard numerical method to simulate\slash integrate the emergent dynamics of the complex system.
Each function has standardised inputs and outputs.


\begin{figure}
\caption{\label{fig:constructPI}The Projective Integration method greatly accelerates simulation\slash integration of a system exhibiting multiple time scales.
The Projective Integration \cref{sec:ProjInt} presents several separate functions, as well as several optional wrapper functions that may be invoked. 
This chart overviews constructing a Projective Integration simulation, whereas \cref{fig:PIchoosemacro} roughly guides which top-level Projective Integration functions should be used.
\cref{sec:ProjInt} fully details each function.}
\centering
\setlength{\WD}{0.05\linewidth}%%%%%%%%%%%%%%%%
\begin{tikzpicture}[node distance = 3ex, auto]
\tikzstyle{bigblock} = [rectangle, draw, thick,   text width=20.5\WD, text badly centered,
    rounded corners, minimum height=4ex]
\tikzstyle{block} = [rectangle, draw=red!80!black, thick, anchor=west, fill=white,
    text width=9.6\WD, text ragged, rounded corners, minimum height=8ex]
 \tikzstyle{smallblock} = [rectangle, draw=red!80!black, thick, anchor=west, fill=white,
    text width=6\WD, text ragged, rounded corners, minimum height=8ex]   
 \tikzstyle{tinyblock} = [rectangle, draw=red!80!black, thick, anchor=west, fill=white,
    text width=4.3\WD, text ragged, rounded corners, minimum height=8ex]    
     \tikzstyle{smallenclose} = [rectangle, draw=red!80!black, thick, anchor=west, fill=white,
    text width=6.2\WD, text ragged, rounded corners, minimum height=8ex]   
     \tikzstyle{refblock} = [rectangle, draw=blue!80!black, thick, anchor=west, fill=blue!5,
    text width=1\WD, rounded corners, minimum height=1.05em] 
\tikzstyle{line} = [draw, -latex']
\tikzstyle{lined} = [draw, latex'-latex']
\node [bigblock,draw=red!80!black,fill=red!10] (gaptooth) {\textbf{Schematic for Projective Integration scheme}

    \begin{tikzpicture}[node distance = 3ex, auto]
    \node [block] (setmicro) {\textbf{Set microsolver}
     
    Code a function that interfaces to your `black-box' microsolver, including the burst time,~\texttt{bT}, of the microsolver. Possible aids:
\begin{itemize}
\item     Use the Patch functions (\cref{fig:constructpatch}) to simulate a large-scale \pde, lattice, etc.
%\item    Use \texttt{bbgen()} if the microsolver is a standard solver, \texttt{ode45} e.g., and needs to be converted into a black-box.
\item    Use \texttt{cmdc()} as a wrapper for the microsolver if the slow variables may change significantly over the microsolver burst.
\end{itemize}};

\node[block, right=2ex of setmicro] (setmacro){\textbf{Set macro simulator}
    
    Set the vector of output times \texttt{tSpan\slash Tspan}, and set initial values~\texttt{x0}.
\begin{itemize}
\item If using \texttt{PIRKn()}, then intervals between times are the projective time-steps.
\item If using \texttt{PIG()}, then intervals between times are as needed by \verb|macroInt|.
    \end{itemize}};
   
\node [smallblock, below=2ex of setmacro] (lift) 
       {\textbf{Set lifting\slash restriction}\quad
    If needed, set functions \texttt{restrict()} and \texttt{lift()} to convert between macro and micro problems\slash variables. These are optional arguments to the Projective Integration functions.};
    
 \node [block, below right=5ex and -9\WD of setmicro] (dopi) {\textbf{Do Projective Integration}\quad 
    Invoke the appropriate Projective Integration function as, e.g., 
    \texttt{x = PIRK2(microBurst, tSpan, x0)}, or
    \texttt{[t,x] = PIG(macroInt, microBurst, Tspan, x0)}. 
    Additional optional outputs inform you of the microscale.};
             \path [line, thick] (setmicro) to[out=-90,in=120] (dopi);
         \path [line, thick] (lift) to[out=180,in=3] (dopi);
         \path [line, thick] (setmacro) to[out=-150,in=30] (dopi);
%          \path [line, thick] (pig) to[out=180,in=0] (dopi);
%          \path [line, thick] (pirk) to[out=180,in=0] (dopi);
%          \draw [dashed, very thick] (pig) -- (pirk);
    \end{tikzpicture}
    };   
\end{tikzpicture}
\end{figure}










\begin{figure}
\centering
\caption{\label{fig:PIchoosemacro}The Projective Integration method greatly accelerates simulation\slash integration of a system exhibiting multiple time scales.
In conjunction with \cref{fig:constructPI}, this chart roughly guides which top-level Projective Integration functions should be used.
\cref{sec:ProjInt} fully details each function.}
\setlength{\WD}{0.081\linewidth}%%%%%%%%%%%%%%%%
\begin{tikzpicture}[node distance = 0.5cm, auto]
\tikzstyle{bigblock} = [rectangle, draw, thick,   text badly centered, 
    text width=12\WD, rounded corners, minimum height=2em]
\tikzstyle{block} = [rectangle, draw=red!80!black, thick, anchor=west, fill=white,
    text width=5\WD, rounded corners, minimum height=4em, text ragged]
 \tikzstyle{smallblock} = [rectangle, draw=red!80!black, thick, anchor=west, fill=white,
    text width=3.4\WD, rounded corners, minimum height=4em, text ragged]  
 \tikzstyle{yesblock} = [rectangle, draw=red!80!black, thick, anchor=west, fill=white,
    text width=1.2\WD, rounded corners, minimum height=1.2em]   
\tikzstyle{line} = [draw, -latex']
\tikzstyle{lined} = [draw, latex'-latex']
\node [bigblock,draw=red!80!black,fill=red!10] (gaptooth) {\textbf{Choosing the macro solver in Projective Integration}\\[2ex]
    \begin{tikzpicture}[node distance = 4ex, auto]
    \node [block,right=5\WD] (timestep) {{Do you know an appropriate time-step for the slow, macroscale, system-level, dynamics?}};
    \node [yesblock, below =2ex of timestep] (timeyes) {Yes};
    \node [yesblock, right=2ex of timestep] (timeno) {No};
    \node [block, below=2ex of timeyes] (slowsol) {{Do you desire a specific method to simulate the slow macroscale dynamics?}};
         \node [yesblock, below =2ex of slowsol] (slowno) {No};
    \node [yesblock,  right=2ex of slowsol] (slowyes) {Yes};
         \node [block, below =2ex of slowno] (pirk) {Simulate\slash integrate with \texttt{PIRK2()}, and maybe confirm with \texttt{PIRK4()}.};
     \node [smallblock, right=10ex of pirk] (pig) 
     {Choose \texttt{PIG()} to simulate\slash integrate};
         \path [line, thick] (timestep) to[out=-90,in=90] (timeyes);
         \path [line, thick] (timestep) to[out=0,in=180] (timeno);
          \path [line, thick] (timeno) to[out=0,in=90] (pig);
          \path [line, thick] (timeyes) to[out=-90,in=90] (slowsol);
          \path [line, thick] (slowsol) to[out=0,in=180] (slowyes);
          \path [line, thick] (slowsol) to[out=-90,in=90] (slowno);
          \path [line, thick] (slowno) to[out=-90,in=90] (pirk);
          \path [line, thick] (slowyes) to[out=0,in=90] (pig);
    \end{tikzpicture}
    };   
\end{tikzpicture}
\end{figure}





\paragraph{Main functions}
\begin{itemize}
\item Projective Integration by second or fourth-order Runge--Kutta is implemented by \verb|PIRK2()| or \verb|PIRK4()| respectively.
These schemes are suitable for precise simulation of the slow dynamics, provided the time period spanned by an application of the microsolver is not too large.

\item Projective Integration with a General method, \verb|PIG()|.
This function enables a Projective Integration implementation of any integration method over macroscale time-steps.
It does not matter whether the method is a standard \script\ algorithm, or one supplied by the user.
\verb|PIG()| should only be used directly in very stiff systems, less stiff systems additionally require \verb|cdmc()|. 

\item \emph{Constraint-defined manifold computing}, \verb|cdmc()|, is a helper function, based on the method introduced in \cite{Gear04},  that iteratively applies the microsolver and backward projection in time.
The result is to project the fast variables close to the slow manifold, without advancing the current time by the burst time of the microsolver.
This function reduces errors related to the simulation length of the microsolver in the \verb|PIG| function.
In particular, it enables \verb|PIG()| to be used on problems that are not particularly stiff.
%\item Black box microsolver generator, \verb|bbgen()|.
%This simple function takes as input a standard solver with a recommended time-step for microscale simulation, and returns a `black-box' microsolver for the Projective Integration functions.
\end{itemize}

The above functions share dependence on a user-specified \emph{microsolver} that accurately simulates some problem of interest. 


The following sections describe the \verb|PIRK2()| and \verb|PIG()| functions in detail, providing an example for each.
The function \verb|PIRK4()| is very similar to \verb|PIRK2()|.
Descriptions for the minor functions follow, and an example using~\verb|cdmc()|.

\input{../ProjInt/PIRK2.m}
\input{../ProjInt/egPIMM.m}
\input{../ProjInt/PIG.m}
\input{../ProjInt/PIRK4.m}
\input{../ProjInt/cdmc.m}

\begin{devMan}
%\input{../ProjInt/bbgen.m}
\input{../ProjInt/PIRKexample.m}
\input{../ProjInt/PIGExample.m}
\input{../ProjInt/PIGExplore.m}



\section{To do/discuss}
\begin{itemize}
\item Implement lifting and restriction for \verb|PIRKn()| functions.

\item Could implement Projective Integration by `arbitrary' Runge--Kutta scheme; that is, by having the user input a particular Butcher table---surely only specialists would be interested.

%\item Can `reverse' the order of projection and microsolver applications with a little fiddling.
%Then output at each user-requested coarse time is the end point of an application of the microsolver---better predictions for fast variables.

\item Can maybe implement microsolvers that terminate a burst when the fast dynamics have settled using, for example, the 'Events' function handle in ode23. 

\item Need projective integration of systems with fast oscillations, perhaps by DMD.

\item Need projective integration for stochastic systems.

\end{itemize}
\end{devMan}


