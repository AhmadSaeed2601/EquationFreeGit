% input *.m files for Projective Integration  AJR, Oct 2017
%!TEX root = ../equationFreeDoc.tex
\section{Projective integration of deterministic ODEs}
\label{sec:ProjInt}
\localtableofcontents

This section provides some good projective integration functions \cite[e.g.]{Gear02b, Gear03c, Givon06}.
The goal is to enable computationally expensive dynamic simulations to be run over long time scales.
Perhaps start by looking at \cref{sec:egPIMM} which codes the introductory example of a long time simulation of the Michaelis--Menton multiscale system of differential equations.

\paragraph{Scenario}
When you are interested in a complex system with many interacting parts or agents, you usually are primarily interested in the self-organised emergent macroscale characteristics.
Projective integration empowers us to efficiently simulate such long-time emergent dynamics.
We suppose you have coded some accurate, fine scale simulation of the complex system, and call such code a microsolver.

The Projective Integration section of this toolbox consists of several functions.
Each function implements over the long-time scale a variant of a standard numerical method to simulate the emergent dynamics of the complex system.
Each function has standardised inputs and outputs.

\paragraph{Main functions}
\begin{itemize}
\item Projective Integration by second or fourth order Runge--Kutta, \verb|PIRK2()| and \verb|PIRK4()| respectively.
These schemes are suitable for precise simulation of the slow dynamics, provided the time period spanned by an application of the microsolver is not too large.
\item Projective Integration with a General solver, \verb|PIG()|.
This function enables a Projective Integration implementation of any solver with macroscale time steps.
It does not matter whether the solver is a standard Matlab or Octave algorithm, or one supplied by the user.
As explored in later examples, \verb|PIG()| should only be used in very stiff systems. 
\end{itemize}
The above functions share dependence on a user-specified `microsolver', that accurately simulates some problem of interest. 
\paragraph{Minor functions}
\begin{itemize}
\item `Constraint-defined manifold computing', \verb|cdmc()|.
This helper function, based on the method introduced in \cite{GearKevrekidis05},  iteratively applies the microsolver and projects the output backwards in time.
The result is to constrain the fast variables close to the slow manifold, without advancing the current time by the duration of an application of the microsolver.
This function can be used to reduce errors related to the simulation length of the microsolver in either the \verb|PIRK| or \verb|PIG| functions.
In particular, it enables \verb|PIG()| to be used on problems that are not particularly stiff.
\item Black box microsolver generator, \verb|bbgen()|.
This simple function takes as input a standard solver with a recommended time step for microscale simulation, and returns a `black box' microsolver for the Projective Integration functions.
\end{itemize}

The following sections describe the \verb|PIRK2()| and \verb|PIG()| functions in detail, providing an example for each.
Descriptions for the minor functions follow, and an example of the use of \verb|cdmc()|.
\verb|PIRK4()| (which is very similar to \verb|PIRK2()|) concludes the section.

\input{ProjInt/egPIMM.m}
\input{ProjInt/PIRK2.m}
\input{ProjInt/PIRK_Example.m}
\input{ProjInt/PIG.m}
\input{ProjInt/PIG_Example.m}

\subsection{Minor functions}
\label{sec:extras}
\input{ProjInt/cdmc.m}
\input{ProjInt/bbgen.m}
\input{ProjInt/PIG_Explore.m}

\input{ProjInt/PIRK4.m}





\subsection{To do/discuss}
\begin{itemize}
\item AJR: inconsistency between description~\(n\) and~\(\ell\), and the code's \verb|N| and~\verb|n|??  Proposed a partial answer.  Also need to be clear what need to be row/column vectors.
\item AJR: is there any advantage to \verb|reshape(t2,[],1)| over \verb|t2(:)|?  Answer appears to be nothing either way.
\item AJR: can this code `integrate' backwards in time using the forward time bursts? Answer: yes. But need to test its accuracy??
\item AJR: replaced \verb|varargout| as I believe many users would be put-off by the extra level of abstraction:  OK? or not?
\item AJR: remember, if used, cater for complex variable simulation by using real transpose~\verb|.'|.
\item could implement Projective Integration by `arbitrary' Runge--Kutta scheme; that is, by having the user input a particular Butcher table---surely only specialists would be interested
\item can `reverse' the order of projection and microsolver applications.
The output at each user-requested coarse time would then be the end point of an application of the microsolver.
\item some kind of minimally invasive checking is needed to ensure the burst length of the microsolver does not make the Projective Integration scheme redundant.
For example, for systems that are not too stiff and for a fourth order Projective Integration scheme PIRK4, we should check that four applications of the microsolver do not bridge the gap between user-specified times.
\item separate subsection for microsolver requirements? 
Then can point to it in each other function.
\item Can maybe implement microsolvers that terminate a burst when the fast dynamics have settles using, for example, the 'Events' function handle in ode23. 
\end{itemize}

