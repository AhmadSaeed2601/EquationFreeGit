% input *.m files for Projective Integration  AJR, Oct 2017
%!TEX root = ../Doc/eqnFreeDevMan.tex
\chapter{Projective integration of deterministic ODEs}
\label{sec:ProjInt}
\localtableofcontents


\section{Introduction}

This section provides some good projective integration functions \cite[e.g.]{Gear02b, Gear03c, Givon06, Maclean2015, Sieber2018}.
The goal is to enable computationally expensive multiscale dynamic simulations\slash integrations to efficiently compute over very long time~scales.

\paragraph{Quick start} 
\cref{sec:pirk2eg} shows the most basic use of a projective integration function.
\cref{sec:egPIMM} shows how to code more variations of the introductory example of a long time simulation of the Michaelis--Menton multiscale system of differential equations.
Then see \cref{fig:constructPI,fig:PIchoosemacro}

\paragraph{Scenario}
When you are interested in a complex system with many interacting parts or agents, you usually are primarily interested in the self-organised emergent macroscale characteristics.
Projective integration empowers us to efficiently simulate such long-time emergent dynamics.
We suppose you have coded some accurate, fine scale simulation of the complex system, and call such code a microsolver.

The Projective Integration section of this toolbox consists of several functions.
Each function implements over a long-time scale a variant of a standard numerical method to simulate\slash integrate the emergent dynamics of the complex system.
Each function has standardised inputs and outputs.


\begin{figure}
\caption{\label{fig:constructPI}The Projective Integration method greatly accelerates simulation\slash integration of a system exhibiting multiple time scales.
The Projective Integration \cref{sec:ProjInt} presents several separate functions, as well as several optional wrapper functions that may be invoked. 
This chart overviews constructing a Projective Integration simulation, whereas \cref{fig:PIchoosemacro} roughly guides which top-level Projective Integration functions should be used.
\cref{sec:ProjInt} fully details each function.}
\centering
\setlength{\WD}{0.05\linewidth}%%%%%%%%%%%%%%%%
\begin{tikzpicture}[node distance = 3ex, auto]
\tikzstyle{bigblock} = [rectangle, draw, thick,   text width=20.5\WD, text badly centered,
    rounded corners, minimum height=4ex]
\tikzstyle{block} = [rectangle, draw=red!80!black, thick, anchor=west, fill=white,
    text width=9.6\WD, text ragged, rounded corners, minimum height=8ex]
 \tikzstyle{smallblock} = [rectangle, draw=red!80!black, thick, anchor=west, fill=white,
    text width=6\WD, text ragged, rounded corners, minimum height=8ex]   
 \tikzstyle{tinyblock} = [rectangle, draw=red!80!black, thick, anchor=west, fill=white,
    text width=4.3\WD, text ragged, rounded corners, minimum height=8ex]    
     \tikzstyle{smallenclose} = [rectangle, draw=red!80!black, thick, anchor=west, fill=white,
    text width=6.2\WD, text ragged, rounded corners, minimum height=8ex]   
     \tikzstyle{refblock} = [rectangle, draw=blue!80!black, thick, anchor=west, fill=blue!5,
    text width=1\WD, rounded corners, minimum height=1.05em] 
\tikzstyle{line} = [draw, -latex']
\tikzstyle{lined} = [draw, latex'-latex']
\node [bigblock,draw=red!80!black,fill=red!10] (gaptooth) {\textbf{Schematic for Projective Integration scheme}

    \begin{tikzpicture}[node distance = 3ex, auto]
    \node [block] (setmicro) {\textbf{Set microsolver}
     
    Define or construct the function \texttt{solver()} that calls a black-box microsolver. Set \texttt{bT}, the time to run microsolver for. Possible aids:
\begin{itemize}
\item     Use the Patch functions (\cref{fig:constructpatch}) to simulate a large-scale \pde, lattice, etc.
%\item    Use \texttt{bbgen()} if the microsolver is a standard solver, \texttt{ode45} e.g., and needs to be converted into a black-box.
\item    Use \texttt{cmdc()} as a wrapper for the microsolver if the slow variables would otherwise change significantly over the microsolver.
\end{itemize}};

    \node[block, right=2ex of setmicro] (setmacro){\textbf{Set macrosolver, define problem}\\[1ex]
        \begin{tikzpicture}
    \node [tinyblock]%, below=0.6cm of pig]
     (pirk) {\textbf{If using \texttt{PIRKn()}:}
      
    Set the vector of output times \texttt{tspan}. Intervals between times are the projective time-steps. Set initial values \texttt{x0}.};
   \node [tinyblock, right=1ex of pirk]%, below right=0.2cm and -1cm of lift]
    (pig) {\textbf{If using \texttt{PIG()}:}
     
    Set the solver \texttt{macro.solver} to be used on the macro scale. Set any needed time inputs or time-step data in \texttt{macro.tspan}. Set initial values \texttt{x0}.};
    \end{tikzpicture}
    };
   
       \node [smallblock, below=2ex of setmacro] (lift) 
       {\textbf{Set lifting\slash restriction}\quad
    If needed, set functions \texttt{restrict()} and \texttt{lift()} to convert between macro and micro problems\slash variables. These are optional arguments to the Projective Integration functions.};
    
 \node [block, below right=5ex and -9\WD of setmicro] (dopi) {\textbf{Do Projective Integration}\quad 
    Invoke the appropriate Projective Integration function as, e.g., 
    \verb|[t,x]=PIRK2(solver,tspan,x0,bT)|, or
    \verb|[t,x]=PIG(solver,macro,x0)|. 
    Additional optional outputs inform you of the microscale.};
             \path [line, thick] (setmicro) to[out=-90,in=120] (dopi);
         \path [line, thick] (lift) to[out=180,in=3] (dopi);
         \path [line, thick] (setmacro) to[out=-150,in=30] (dopi);
%          \path [line, thick] (pig) to[out=180,in=0] (dopi);
%          \path [line, thick] (pirk) to[out=180,in=0] (dopi);
%          \draw [dashed, very thick] (pig) -- (pirk);
    \end{tikzpicture}
    };   
\end{tikzpicture}
\end{figure}










\begin{figure}
\centering
\caption{\label{fig:PIchoosemacro}The Projective Integration method greatly accelerates simulation\slash integration of a system exhibiting multiple time scales.
In conjunction with \cref{fig:constructPI}, this chart roughly guides which top-level Projective Integration functions should be used.
\cref{sec:ProjInt} fully details each function.}
\setlength{\WD}{0.081\linewidth}%%%%%%%%%%%%%%%%
\begin{tikzpicture}[node distance = 0.5cm, auto]
\tikzstyle{bigblock} = [rectangle, draw, thick,   text badly centered, 
    text width=12\WD, rounded corners, minimum height=2em]
\tikzstyle{block} = [rectangle, draw=red!80!black, thick, anchor=west, fill=white,
    text width=5\WD, rounded corners, minimum height=4em, text ragged]
 \tikzstyle{smallblock} = [rectangle, draw=red!80!black, thick, anchor=west, fill=white,
    text width=3.4\WD, rounded corners, minimum height=4em, text ragged]  
 \tikzstyle{yesblock} = [rectangle, draw=red!80!black, thick, anchor=west, fill=white,
    text width=1.2\WD, rounded corners, minimum height=1.2em]   
\tikzstyle{line} = [draw, -latex']
\tikzstyle{lined} = [draw, latex'-latex']
\node [bigblock,draw=red!80!black,fill=red!10] (gaptooth) {\textbf{Choosing the macro solver in Projective Integration}\\[2ex]
    \begin{tikzpicture}[node distance = 4ex, auto]
    \node [block,right=5\WD] (timestep) {{Is an appropriate time-step known for the slow dynamics?}};
    \node [yesblock, below =2ex of timestep] (timeyes) {Yes};
    \node [yesblock, right=2ex of timestep] (timeno) {No};
    \node [block, below=2ex of timeyes] (slowsol) {{Is a specific solver desired to simulate the slow dynamics?}};
     \node [smallblock, below right=2ex and 2ex of timeno] (pig) {Choose \texttt{PIG()} to simulate\slash integrate};
         \node [yesblock, below =2ex of slowsol] (slowno) {No};
    \node [yesblock,  right=2ex of slowsol] (slowyes) {Yes};
         \node [block, below =2ex of slowno] (pirk) {Choose \texttt{PIRK2()} or \texttt{PIRK4()} to  simulate\slash integrate};
         \path [line, thick] (timestep) to[out=-90,in=90] (timeyes);
         \path [line, thick] (timestep) to[out=0,in=180] (timeno);
          \path [line, thick] (timeno) to[out=0,in=90] (pig);
          \path [line, thick] (timeyes) to[out=-90,in=90] (slowsol);
          \path [line, thick] (slowsol) to[out=0,in=180] (slowyes);
          \path [line, thick] (slowsol) to[out=-90,in=90] (slowno);
          \path [line, thick] (slowno) to[out=-90,in=90] (pirk);
          \path [line, thick] (slowyes) to[out=0,in=-90] (pig);
    \end{tikzpicture}
    };   
\end{tikzpicture}
\end{figure}





\paragraph{Main functions}
\begin{itemize}
\item Projective Integration by second or fourth-order Runge--Kutta, \verb|PIRK2()| and \verb|PIRK4()| respectively.
These schemes are suitable for precise simulation of the slow dynamics, provided the time period spanned by an application of the microsolver is not too large.

\item Projective Integration with a General solver, \verb|PIG()|.
This function enables a Projective Integration implementation of any solver with macroscale time-steps.
It does not matter whether the solver is a standard Matlab algorithm, or one supplied by the user.
As explored in later examples, \verb|PIG()| should only be used in very stiff systems. 

\item `Constraint-defined manifold computing', \verb|cdmc()|.
This helper function, based on the method introduced in \cite{Gear04},  iteratively applies the microsolver and projects the output backwards in time.
The result is to constrain the fast variables close to the slow manifold, without advancing the current time by the duration of an application of the microsolver.
This function can be used to reduce errors related to the simulation length of the microsolver in either the \verb|PIRK| or \verb|PIG| functions.
In particular, it enables \verb|PIG()| to be used on problems that are not particularly stiff.
%\item Black box microsolver generator, \verb|bbgen()|.
%This simple function takes as input a standard solver with a recommended time-step for microscale simulation, and returns a `black-box' microsolver for the Projective Integration functions.
\end{itemize}

The above functions share dependence on a user-specified `microsolver', that accurately simulates some problem of interest. 


The following sections describe the \verb|PIRK2()| and \verb|PIG()| functions in detail, providing an example for each.
Then \verb|PIRK4()| is very similar to \verb|PIRK2()|.
Descriptions for the minor functions follow, and an example of the use of~\verb|cdmc()|.

\input{../ProjInt/PIRK2.m}
\input{../ProjInt/egPIMM.m}
\input{../ProjInt/PIG.m}
\input{../ProjInt/PIRK4.m}
\input{../ProjInt/cdmc.m}

\begin{devMan}
%\input{../ProjInt/bbgen.m}
\input{../ProjInt/PIRK_Example.m}
\input{../ProjInt/PIGExample.m}
\input{../ProjInt/PIGExplore.m}



\section{To do/discuss}
\begin{itemize}
\item Can we implement for Octave?  We would like to use nested functions for some examples, because the function code then inherits parameter(s) from the main function.  However, in Octave we cannot then use handles to these nested functions due to the error ``handles to nested functions are not yet supported"---which apparently is not going to be fixed anytime soon (as at March 2019).

\item could implement Projective Integration by `arbitrary' Runge--Kutta scheme; that is, by having the user input a particular Butcher table---surely only specialists would be interested.

\item can `reverse' the order of projection and microsolver applications with a little fiddling.
Then output at each user-requested coarse time is the end point of an application of the microsolver---better predictions for fast variables.

\item Can maybe implement microsolvers that terminate a burst when the fast dynamics have settled using, for example, the 'Events' function handle in ode23. 

\item Need projective integration of systems with fast oscillations, perhaps by DMD.

\item Need projective integration for stochastic systems.

\end{itemize}
\end{devMan}


